%% LyX 2.1.3 created this file.  For more info, see http://www.lyx.org/.
%% Do not edit unless you really know what you are doing.
\documentclass[english]{article}
\renewcommand{\familydefault}{\sfdefault}
\usepackage[T1]{fontenc}
\usepackage[latin9]{inputenc}
\usepackage{geometry}
\geometry{verbose,tmargin=1in,bmargin=1in,lmargin=1in,rmargin=1in}
\setlength{\parskip}{\medskipamount}
\setlength{\parindent}{0pt}
\usepackage{amssymb}

\makeatletter
%%%%%%%%%%%%%%%%%%%%%%%%%%%%%% User specified LaTeX commands.
\usepackage{amsmath}
\date{}

\makeatother

\usepackage{babel}
\begin{document}

\title{Small Problem 1: Bayesian Linear Regression}

\maketitle
\vspace{-0.5in}


\section*{Summary}


\subsection*{Given:}

A set of training points $\left\{ \left(x_{i},y_{i}\right)\right\} _{i=1}^{N}$,
where $x_{i}\in\mathbb{R}^{d}$ and $y_{i}\in\mathbb{R}$

Generative model for the data in terms of the weight vector $w$ and
hyperparameters, as described below.


\subsection*{Find:}

Query 1: The posterior probability distribution of $w$.


\subsection*{Metrics:}

Metric 1: Expected squared Euclidean distance between the predicted
mean $\hat{w}$ and the true mean $w$, where the expectation is taken
with respect to the posterior distribution.

Metric 2: Total variation distance between the computed posterior
and the correct posterior over $w$.


\section*{Details}

The file ``problem-1-generator.R'' contains R code to generate the
true regression coefficients and the input training data. The model
is 
\begin{eqnarray*}
\Sigma_{1} & = & 2\mathbf{I}_{5\times5}\\
\mu & \sim & \mathcal{N}\left(0,\Sigma_{1}\right)\\
\Sigma_{2} & = & \mathbf{I}_{5\times5}\\
\Sigma_{\mbox{prior}} & \sim & \mbox{Wishart}\left(1,\Sigma_{2}\right)\\
w & \sim & \mathcal{N}\left(\mu,\Sigma_{\mbox{prior}}^{-1}\right)\\
x_{ij} & \sim & \mbox{Uniform}\left(-1,1\right)\\
\tau & \sim & \mbox{Gamma}\left(0.5,2\right)\\
\epsilon_{i} & \sim & \mathcal{N}\left(0,\frac{1}{\tau}\right)\\
y_{i} & = & \sum_{j}x_{ij}w_{j}+\epsilon_{i}
\end{eqnarray*}


The file contains 500 training examples generated from a single run
of the R code. There are four covariates generated uniformly from
$\left[-1,1\right]$. The values of the variables that generated the
data are
\begin{eqnarray*}
\mu & = & (-1.8195312,1.2237587,0.8361809,-2.6017006,-2.3574193)\\
\Sigma_{\mbox{prior}} & = & \mbox{(see ``problem-1-prior.Sigma.csv'')}\\
w & = & (-1.731855,2.986017,2.698284,-3.591651,-3.714157)\\
\left(x_{i},y_{i}\right) & = & \mbox{(see ``problem-1-data.csv'')}
\end{eqnarray*}


Queries/Metrics: 
\begin{enumerate}
\item Let $P\left(\hat{w}|D\right)$ be the posterior distribution of the
estimated weight vector. One metric is the expected squared error
$\mathbb{E}\left[\left\Vert \hat{w}-w\right\Vert ^{2}\right]$ under
this distribution.
\item We have provided samples generated from the true posterior distribution
$P_{\mbox{true}}\left(\hat{w}|D\right)$. We can estimate the total
variation distance between the true distribution and your estimate
$P\left(\hat{w}|D\right)$ using the samples generated by your estimated
distribution: 
\[
\int_{w}\left|P\left(\hat{w}|D\right)-P_{\mbox{true}}\left(\hat{w}|D\right)\right|dw
\]
\end{enumerate}

\end{document}
